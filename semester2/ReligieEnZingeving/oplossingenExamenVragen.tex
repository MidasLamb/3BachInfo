\documentclass[11pt,a4paper,titlepage]{article}
\usepackage[utf8]{inputenc}
\usepackage[dutch]{babel}

\usepackage{color}
\usepackage{hyperref}

\title{Religie en zingeving oplossing examenvragen}
\author{Pieter-Jan Coenen}
\date{April 2017}


\begin{document}

\maketitle
\newpage
\tableofcontents
\newpage

\section{Kernbegrippen}
\subsection{Pale Blue Dot}
Pale Blue Dot is een foto van het heelal die genomen is door de ruimtesonde Voyager 1 na zijn missie op vraag van Carl Sagan. Op de foto is de aarde te zien op afstand van enkele miljarden kilometers. De aarde is op de foto nog maar een zeer klein blauw puntje in het gigantisch grootte heelal tussen alle andere sterren.\\
Sagen heeft ook zelf deze afbeelding becommentarieerd, het feit dat de aarde maar zo'n klein puntje is in het gigantische heelal deed bij hem bijvoorbeeld de vraag reizen waarom er wel een god geïnteresseerd zou zijn in dat super klein puntje (de aarde). Maar ook andere vragen zoals ``Wat moet de rol van de wetenschap zijn?'', ``Wat moet de rol van het geloof zijn?''  en ``Zou er toch meer zijn, zou er toch een God bestaan?''.
\subsection{Theïsme en deïsme}
Deïsme houdt in dat God de schepper van het universum en/of de aarde is, maar dat hij sinds de schepping op geen enkele wijze nog ingrijpt (God als 'horlogemaker').\\ Bij het theïsme daarentegen heeft God de aarde geschapen heeft en is deze na de schepping nog steeds betrokken.
\subsection{Theïstische evolutie (Collins)}
Collins is een verdediger van de theïstische evolutie, hij gelooft dat God de natuurwetten ingesteld heeft en dat de evolutie de manier is waarop hij zijn schepping realiseert.
\subsection{William Paley}
In zijn boek Natural Theology argumenteerde Paley dat er een God of toch een intelligent designer moet bestaan en dat de aanwijzigen hiervoor waren terug te vinden in de natuur. Hij vergelijkt de natuur met een uurwerk waarvan alle radartjes precies op elkaar afgestemd zijn. Bijvoorbeeld het oog van de mens zit zeer complex in elkaar.  Ook in de fysica speelt alles perfect op elkaar in. Uit het feit dat alles in de natuur zo perfect op elkaar is afgestemd besluit hij dat er wel een soort van intelligent designer moet bestaan.
\subsection{Natuurlijke theologie }
Natuurlijke theologie is het beargumenteren van het bestaan van een God m.b.v. aanwijzigingen uit de natuur, zoals bijvoorbeeld William Paley deed. In Paley's boek "Natural Theologie" vergelijkt hij de natuur met uurwerk, waarvan alle radartjes precies op elkaar afgestemd zijn. Bijvoorbeeld het oog van de mens zit zeer complex in elkaar.  Ook in de fysica speelt alles perfect op elkaar in. Omwille van deze complexe/vernuftige structuur van de natuur moet er wel een intelligent designer bestaan. \\ Darwin haalde later deze argumenten onderuit door de ontdekking van de natuurlijke selectie, de natuurlijke selectie bewijst dat de complexiteit gewoon gegroeid is.
\subsection{NOMA-principe}
NOMA = “non-overlapping magisteria” (Stephen J. Gould): de visie dat geloof en wetenschap twee onderscheiden domeinen zijn met eigen taak en autoriteit (wetenschap verklaart de werkelijkheid; geloof is bezig met levensvragen). Er kan daarom geen conflict zijn tussen beide.
\subsection{Wijsheid (volgens Stephen Jay Gould)}
Volgens Stephen Jay Gould is wijsheid aandacht geven aan zowel geloof als wetenschap. Voor Gould is er een scheiding tussen wetenschap en geloof. Maar kan je de echte wijsheid enkel bekomen door naar beide te kijken, we zijn pas wijs als we ze allebei kunnen integreren.
\subsection{Procrustesbed (Taede Smedes)}
Volgens Taede Smedes vormt de hedendaagse integratie van wetenschap en geloof (dus het feit dat ze met elkaar kunnen verzoend worden) vrijwel altijd een soort procrust[e]sbed vormt, waarbij of de theologie of natuurwetenschap op maat wordt gesneden om compatibiliteit met de andere helft te garanderen.
\section{Synthesevragen}
\subsection{Bespreek de drie klassieke modellen om de verhouding tussen geloof/religie en wetenschap te denken. }
\begin{enumerate}
\item Het conflictmodel stelt dat geloof en wetenschap concurrerende
alternatieven zijn. Je kan niet beide tegelijk aanhouden en moet dus
kiezen tussen beide. Dit is de positie van militante atheïsten zoals
Richard Dawkins, maar ook van aanhangers van het creationisme. 
\item Het kloofmodel probeert dit conflict op de lossen door te zeggen dat
geloof en wetenschap totaal verschillende benaderingen van de
werkelijkheid zijn die zo grondig verschillen dat ze niet met elkaar in
conflict kunnen komen. Zo verdedigt Stephen Jay Gould het NOMA-principe:
wetenschap beschrijft de werkelijkheid (hoe-vragen) en religie
houdt zich bezig met zinvragen (waarom-vragen). Probleem is dat
religie hier wel de zwaarste prijs betaalt (wetenschap bepaalt wat
religie nog kan zeggen) en het is nog maar de vraag of religie in
isolement kan blijven van de wetenschap. 
\item Het harmoniemodel probeert het conflict tussen geloof en wetenschap op te lossen door beide te integreren en ze in elkaar of in een overkoepelende eenheidstheorie te laten opgaan. Een verdediger hiervan is bijvoorbeeld Francis Collins die verdedigt dat God de natuurwetten ingesteld heeft en dat de evolutie de manier is waarop God zijn schepping realiseert. Volgens Taede Smedes leidt het harmoniemodel echter vaak tot nieuwe conflicten omdat de één op maat van de ander gesneden wordt om ze samen te laten passen. 
\end{enumerate}
\subsection{Hoe brengt Francis Collins geloof en wetenschap samen? Welke kritiek kan je op Collins
geven met behulp van Frans de Waals TED-lezing? Op welke manier bevestigt de
mogelijkheid van deze kritiek de visie van Taede Smedes op het probleem met integratie van
geloof en wetenschap? }
Francis Collins is een voorstander van het harmoniemodel, hij probeert het conflict tussen geloof en wetenschap dus op te lossen door beide in elkaar te integreren m.b.v. een overkoepelde eenheidstheorie. Hij zegt dat God de natuurwetten ingesteld heeft en dat de evolutie de manier is waarop God zijn schepping realiseert, hij is dus een voorstander van de theïstische evolutie. Het beoefenen van wetenschap is dus in zijn ogen een manier om meer inzicht te krijgen in de schepping van God, een inkijk in de geest van God. Hij beschouwd het beoefenen van wetenschap zelfs als een soort van Gods aanbidding. Door aan wetenschap te doen kom je volgens zijn theorie immers meer te weten over de schepping van God, want voor de theïstische evolutionisten heeft God de evolutie en de wetten van de natuur geschapen. Je kan de diepere waarheid bekomen door geloof en wetenschap te combineren. \\
\textcolor{red}{Nog aan te vullen.}
\subsection{Bespreek de visie van Stephen Jay Gould op de verhouding tussen religie en wetenschap.
Wat wil Goud met zijn voorstel bereiken? Slaagt hij in zijn opzet (waarom wel/niet)?}
Stephen Jay Gould is een voorstander van het kloofmodel, wat wil zeggen dat volgens hem geloof en wetenschap totaal verschillende benaderingen van de werkelijkheid zijn die zo grondig verschillen dat ze niet met elkaar in conflict kunnen komen. Hij is de grondlegger van het NOMA-principe (“non-overlapping magisteria”). Het is de visie dat geloof en wetenschap twee verschillende domeinen zijn met eigen taak en autoriteit, wetenschap beschrijft de werkelijkheid (hoe-vragen, de feiten, de theorie) en religie houdt zich bezig met zinvragen (waarom-vragen). Omwille van die redenen suggereert Gould dat er geen conflict kan zijn tussen beide. Toch slaagt hij niet volledig in zijn opzet, omdat het probleem van deze houding is dat religie hier de zwaarste prijs betaalt (wetenschap bepaalt wat religie nog kan zeggen) en het is de vraag of religie in isolement kan blijven van de wetenschap. Zoals Taede Smedes stelt, vormt de integratie van wetenschap en geloof een procrustesbed, waarbij de theologie of de wetenschap steeds op maat wordt gesneden met het andere om compatibliteit te garanderen.\\ \textcolor{red}{Nog aan te vullen.}
\section{Stellingen}
\subsection{Wereldwijd hanteren de meeste wetenschappers het conflictmodel om de relatie tussen
religie en wetenschap te denken. }
\textcolor{red}{\textbf{Fout}}\\\\
Volgens de studie en de cijfers die we in de les hebben overlopen hanteren wereldwijd de meeste wetenschappers het kloofmodel om de relatie tussen religie en wetenschap te denken.
\subsection{Wereldwijd worden wetenschappers minder religieus door het beoefenen van hun
wetenschap.}
\textcolor{red}{\textbf{Fout}}\\\\
Volgens de studie en de cijfers die we in de les hebben overlopen voor de verschillende landen, claimde maximaal 22\% van de wetenschappers dat ze minder religieus werden door het beofenen van wetenschap, m.a.w. vond een grote meerderheid van de wetenschappers niet dat ze minder religieus werden door het beoefenen van wetenschap.
\subsection{Wetenschappers zijn altijd minder religieus dan de doorsneebevolking in hun land.}
\subsection{Het conflictmodel staat het sterkst in contexten waarin het monotheïsme dominant is (of
was).}
\subsection{Wereldwijd verdedigen de meeste wetenschappers een scheiding tussen religie en wetenschap.}
\textcolor{green}{\textbf{Juist}}\\\\
Volgens de studie en de cijfers die we in de les hebben overlopen hanteren wereldwijd de meeste wetenschappers het kloofmodel en zijn ze dus voor een scheiding tussen religie en wetenschap.
\subsection{Wetenschappers kunnen aanhanger zijn van het kloofmodel omdat ze er eigenlijk van uitgaan dat er een onoplosbaar conflict is tussen beide.}
\textcolor{green}{\textbf{Juist}}\\\\
Het is inderdaad mogelijk dat een wetenschapper een aanhanger is van het kloofmodel, juist omdat hij/zij vindt dat er een conflict is tussen beide.
\subsection{Francis Collins’ integratie van geloof en wetenschap vooronderstelt eigenlijk een voorafgaande scheiding tussen beide.}
\textcolor{red}{\textbf{Fout}}\\\\
Francis Collins verdedigt de theologische evolutie, dit betekend dat God de natuurwetten ingesteld heeft en dat de evolutie de manier is waarop hij zijn schepping realiseert. Aangezien hij stelt dat God zelf de wetten van de natuur heeft geschapen, is er geen onderscheid tussen wetenschap en geloof.
\subsection{Stephen Jay Gould verdedigt een scheiding tussen geloof en wetenschap maar eigenlijk is het NOMA-principe een verdoken vorm van conflictmodel.}
\textcolor{red}{\textbf{Fout}}\\\\
Integendeel, het NOMA-principe is de visie dat geloof en wetenschap twee verschillende domeinen zijn met eigen taak en autoriteit (wetenschap verklaart de werkelijkheid; geloof is bezig met levensvragen). Er kan daarom geen conflict zijn tussen beide.
\subsection{Wie vandaag een harmonie tussen geloof en natuurwetenschap nastreeft, bevordert volgens
Taede Smedes eigenlijk het conflict tussen beide.}
\textcolor{green}{\textbf{Juist}}\\\\
Om de integratie van geloof en wetenschap mogelijk te maken, moet de één op maat gesneden worden van de ander (of beide op maat van elkaar) (cf. het beeld van het procrustesbed). Hierdoor komen ze in elkaars vaarwater en ontstaan nieuwe conflicten. 
\section{Korte open vragen}
\subsection{Waarom zijn we de colleges begonnen met de reflectie van Carl Sagan over Pale Blue Dot?}
De colleges zijn begonnen met de reflectie van Carl Sagan over Pale Blue Dot, omdat de afbeelding aantoont dat wij maar een zeer klein puntje zijn in een gigantisch heelal waar heel veel niet over geweten is.\\ Dit doet bij Sagan ook een hele boel vragen reizen "Waarom zou een God in ons geïnteresseerd zijn?", "Wat is nu de zin van het leven?", "Hoe komt het heelal er?", "Wat is de rol van wetenschap?", "Maak wetenschap ons gelukkiger?".\\ De video was dus een perfecte keuze om de lessen te starten omdat ze ons doet stilstaan en ons nieuwsgierig maakt naar de essentiele vragen uit deze cursus.
\subsection{In 1986 publiceerde Richard Dawkins een boek met als titel De blinde horlogemaker. Wat is de betekenis van deze titel?}
De titel van het boek verwijst naar natuurlijke selectie.  William Paley schreef in zijn boek Natural Theology, dat het bestaan van een complex wezen zoals de mens het bestaan van een schepper bevestigt, net zoals het bestaan van een complex horloge het bestaan van een horlogemaker bevestigt. Dawkins haalt in zijn boek de theorie van Paley onderuit, omdat de mens complex is geworden door natuurlijke selectie, door de willekeurige mutaties en de cumulatieve selectie daarop. De natuur is dus eigenlijk een blinde horlogemaker die zonder enig inzicht of planning een horloge schept.
\subsection{Wat is de parabel van de onzichtbare tuinman en hoe functioneerde zij binnen de cursus?}
De parabel van de onzichtbare tuinman gaat over twee mannen die na lange tijd terug keren naar hun huis en opmerken dat hun tuin er nog steeds onderhouden uitziet. Eén van de twee mannen is er dan ook van overtuigd dat er een tuinman is die hun tuin onderhoudt als ze er niet zijn, tewijl de andere man denkt dat dit niet het geval is. Om te weten wie nu gelijk heeft doen ze een hoop testen, maar niets biedt uitsluitsel. De ene man heeft argumenten waarom de tuinman wel bestaat en de andere heeft argumenten waarom de tuinman niet bestaat. Het is een onoplosbare discussie, want beide mannen kijken met een andere bril. \\
In de curusus komt dit verhaal terug om aan te tonen dat het moeilijk is om over geloof te discusiëren en dat je het bestaan van God niet kan bewijzen, net zoals het bestaan van de tuinman niet kan worden bewezen. En dat het daarom moeilijk is om over het bestaan van God te discusiëren.
\subsection{Waarom hebben we het tijdens de colleges over de sluipwespen gehad?}
Sluipwespen zijn wespen die rupsen een gif toedienen en er vervolgens hun eitjes in de rups leggen. Het gif zorgt ervoor dat de rups gebrainwashed is en de eitjes van de wesp verdedigt, de rups zal ook sterven nadat de larven uit de eitjes zijn gekropen. \\ 
Het is dus vrij gruwelijk gedrag van de natuur, juist daarom claimed Gould dat we de natuur niet als basis mogen nemen voor onze moraal. De natuur is soms gruwelijk en daarom hebben we geloof nodig. Dit is één van de redenen waarom Gould pleit voor een scheiding tussen geloof en wetenschap.
\subsection{Wat wordt bedoeld met “resonanties” tussen geloof en wetenschap? }
 “Resonantie” is een term uit de muziekwereld (eigenlijk uit de natuurkunde) en betekent zoveel als
“meetrillen” of “samen klinken”. Hier wordt de term gebruikt om aan te
geven dat er tussen geloof en wetenschap op een bepaald punt een
zekere verwantschap gevoeld kan worden, maar ook niet meer dan dat
\subsection{Hoe functioneerde het fragment uit Tarkovski’s Nostalghia in de opbouw van de colleges en wat hebben we er uit geleerd?}
Het fragment gaat over Eugenia die niet kan knielen, maar wel wil. Ze zou wel willen geloven, maar ze kan het niet. De vraag is of ze gokt op het bestaan van God (en tijdens haar leven even een effort doet) en eeuwig geluk na het leven heeft of dat ze ervan uitgaat dat God niet bestaat en later eeuwig ongelukkig is als hij toch zou bestaan. Ze is dus eigenlijk opzoek naar geluk. Dit fragement maakt de overgang naar het volgende onderwerp in de cursus, namelijk geluk. Wereldwijd vinden mensen geluk belangrijk, maar de vraag is of geluk wel het belangrijkste is. Nog bijkomende vragen zijn dan of wetenschap ons gelukkig kan maken en of het eigenlijk wel goed is om gelukkiger te willen worden.
\end{document}