\documentclass[12pt]{article}
\usepackage[utf8]{inputenc}
\usepackage[dutch]{babel}
\usepackage[dutch]{babelbib}
\usepackage{amsmath}
\renewcommand{\familydefault}{\sfdefault}

\title{AB: Oefenzitting 1}
\author{Stijn Caerts, Pieter-Jan Coenen}
\date{6 oktober 2016}

\begin{document}
    \maketitle
    
    \section{Oefening 1}
    Voor de taal $L$ bestaat er een NFA. Als we alle bogen van richting veranderen en de eindtoestand veranderen in de begintoestand en omgekeerd, verkrijgen we een NFA die overeenstemt met $L^{R}$. Aangezien een NFA steeds een reguliere taal bepaalt, is $L^{R}$ regulier. 
    $$ NFA_1 = \langle Q, \Sigma, \delta, q_s, F \rangle $$
    $$ NFA_2 = \langle Q \cup \{ q_e \}, \Sigma, \delta^{'}, q_s, \{q_e\} \rangle $$
    
    $$ \forall q \in F: \delta^{'}(q,\epsilon) = \{q_e\} $$
    
    $$ NFA_R = \langle Q_1, \Sigma, \delta^{''}, q_e, \{q_b\} \rangle $$
    $$ \forall q_i \in Q, \forall \sigma \in \Sigma: \delta^{''}(q_i,\sigma) = \{q_j \in Q | q_i \in \delta^{'}(q_j,\sigma)\} $$
    
    of aan de hand van reguliere expressies:
    \begin{itemize}
	\item $ \phi^R = \phi $
	\item $ \epsilon = \epsilon $
	\item $ a^R = a $ met $ a \in \Sigma $
	\item $ (E_1 E_2)^R = E_2^R E_1^R $
	\item $ (E_1 | E_2)^R = E_1^R | E_2^R $
	\item $ (E^*)^R = (E^R)^* $
    \end{itemize}

    
    
    \section{Oefening 2}
    \subsection{a)}
    \begin{itemize}
        \item Reguliere expressie: $ 1 ((0|1)1)^{*} (0|1|\epsilon) $ of $ (1(0|1))^{*} 1 | (1(0|1))^{*} $
    \end{itemize}
    
    \subsection{b)}
    \begin{itemize}
        \item Reguliere expressie: $ (0)^{*} (001|100|010|00) (0)^{*} $
    \end{itemize}
    
    \subsection{c)}
    \begin{itemize}
        \item Reguliere expressie: $ ((0|1)^{*}0(0|1)^{*})|1|11111^{*}|\epsilon $
    \end{itemize}
    
    \subsection{d)}
    \begin{itemize}
        \item triviale gevallen: enkel $0$'en, $1$'en of $\epsilon$
        \item als je begint met een 1, moet je ook eindigen met een 1
        \item als je begint met een 0, moet je ook eindigen met een 0
        \item Reguliere expressie:  $ (1 (0|1)^{*} 1) | (0 (0|1)^{*} 0) | (1|0|\epsilon)$
    \end{itemize}
    
    \section{Oefening 3}
    \subsection{a)}
    \begin{itemize}
        \item Reguliere expressie: $ (a^{n})^{*} $
        \item Automaat: n bogen met a tot eindtoestand, $\epsilon$-boog terug naar begintoestand. Als je nul ziet als een veelvoud van $n$, dan is de begintoestand ook een eindtoestand.
    \end{itemize}
    
    \subsection{b)}
    \begin{itemize}
        \item $n$ toestanden die de rest bij deling door $n$ voor stellen.
        \item Transitiefunctie: $ \delta (q_k, i) = (2k + i) \mod n $
    \end{itemize}
\end{document}